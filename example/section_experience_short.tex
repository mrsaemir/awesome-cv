% Awesome Source CV LaTeX Template
%
% This template has been downloaded from:
% https://github.com/darwiin/awesome-neue-latex-cv
%
% Author:
% Christophe Roger
%
% Template license:
% CC BY-SA 4.0 (https://creativecommons.org/licenses/by-sa/4.0/)

%Section: Work Experience at the top
\sectionTitle{Expériences Professionelles}{\faSuitcase}
%\renewcommand{\labelitemi}{$\bullet$}
\begin{experiences}
  \experience
    {Oct 2024} {Internet Researcher}{\link{https://www.mpi-inf.mpg.de/departments/inet}{Max-Planck-Institut}}{Germany}
    {Apr 2024} {
      During my time at MPI, I was tasked with analyzing a massive dataset of 3.5 billion DNS TXT records to identify insecure or malicious use cases. This project aimed to replicate the findings of the paper “TXTing 101: Finding Security Issues in the Long Tail of DNS TXT Records” by Van Der Toorn et al. (2020) and extend the analysis using a different dataset. There, my primary responsibilities included analyzing and categorizing unique DNS TXT records and investigating abnormal use cases.
    }
    {Python, Apache Spark, Apache Hadoop}
  \emptySeparator
  \experience
    {Now} {Software Engineer}{\link{https://geops.com}{geOps}}{Germany}
    {July 2023} {
      At geOps, I contributed to developing and enhancing multiple Python repositories as part of a collaborative team. My primary responsibilities included maintaining and expanding these repositories, with a particular emphasis on data pipelines. In this capacity, I ensured the smooth operation of data pipelines, handling the extraction, transformation, and loading of data. This role provided me with valuable experience working with geospatial and real-time data in the public transport sector. Additionally, I worked on upgrading existing repositories to use SQLAlchemy2 and integrating its asynchronous drivers fo numerous components. I also contributed to the integration of various transportation APIs from providers such as Deutsche Bahn, Swiss Railways, S-Bahn Munich, and others.    
    }
    {Python, Asyncio, Django, Flask, SQLAlchemy, GraphQL, PostgreSQL, Redis, GTFS/GTFS-RT}
  \emptySeparator
  \experience
  {Mar 2023} {Software Engineer}{\link{https://gridsingularity.com/}{GridSingularity}}{Germany}
  {Apr 2022} {
    GridSingularity’s platform enables the creation of peer-to-peer, grid-aware energy marketplaces, including energy communities, that facilitate the exchange of energy between producers and consumers in a secure and transparent manner. At GSy, I was responsible for further developing the energy exchange, the backend, and the client SDKs using the Python programming language, a set of different database and messaging technologies. The work on the backend included:
  }
  {Python, Django, GraphQL, PostgreSQL, DRF}
  \emptySeparator
  \experience
  {Mar 2022} {Software Engineer}{\link{https://www.ctatuscolana.it/}{Tuema}}{Italy}
  {Jan 2020} {
    At Tuema I was responsible for developing and maintaining the backend and the CI/CD pipelines for the PlayViaggi project. The work on the backend included designing an integrated accounting system and migrating existing Django Template to REST APIs while refactoring legacy code.
  }
  {Python, Django, PostgreSQL, DRF, Docker}
  \emptySeparator
  \experience
  {Jan 2020} {Software Engineer Intern}{Mahsan}{Iran}
  {Apr 2019} {
    During my internship at Mahsan, I was responsible for developing web APIs that allowed users to interact a DataDiode device. A DataDiode functions similarly to a firewall but provides more guarantees regarding one-way data flow. My primary focus at Mahsan was to create APIs that could modify settings across various system components, including the operating system, DD engine, OpenLDAP, network interfaces, NTP, Active Directory, and a range of other system-level settings.
  }
  {Python, Django, Linux}
  \emptySeparator
\end{experiences}
